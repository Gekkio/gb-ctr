\chapter{MBC1 mapper chip}

The majority of games for the original Game Boy use the MBC1 chip. MBC1
supports ROM sizes up to 16 Mbit (128 banks of \texttt{\$4000} bytes) and RAM
sizes up to 256 Kbit (4 banks of \texttt{\$2000} bytes). The information in
this section is based on my MBC1 research, Tauwasser's research notes
\cite{tauwasser_mbc1}, and Pan Docs \cite{pandocs}.

\section{MBC1 registers}

\begin{caveat}
  These registers don't have any standard names and are usually referred to
  using their address ranges or purposes instead. This document uses names to
  clarify which register is meant when referring to one.
\end{caveat}

The MBC1 chip includes four registers that affect the behaviour of the chip.
The only cartridge bus address signals connected to the MBC are A13-A15, so
lower address bits don't matter when the CPU is accessing the MBC and all
registers are effectively mapped to address ranges instead of single addresses.
All registers are smaller than 8 bits, and unused bits are simply ignored
during writes. The registers are not directly readable.

\begin{register}[H]
  \caption{\texttt{\$0000-\$1FFF} - RAM\_EN - MBC1 RAM enable register}
  {
    \ttfamily
    \begin{tabu} to \textwidth {|X[c]|X[c]|X[c]|X[c]|X[c]|X[c]|X[c]|X[c]|}
      \everyrow{\hline}
      \hline
      U                     & U                     & U                     & U                     & W-0                                    & W-0 & W-0 & W-0   \\
      \cellcolor{LightGray} & \cellcolor{LightGray} & \cellcolor{LightGray} & \cellcolor{LightGray} & \multicolumn4{c|}{RAM\_EN<3:0>} \\
      \rowfont{\rmfamily\small}
      bit 7                 & 6                     & 5                     & 4                     & 3                                      & 2   & 1   & bit 0 \\
      \hline
    \end{tabu}
  }

  \begin{description}[leftmargin=5em, style=nextline]
    \item[bit 7-4]
      \textbf{Unimplemented}: Ignored during writes
    \item[bit 3-0]
      \textbf{RAM\_EN<3:0>}: RAM enable register \\
      \texttt{1010=} enable access to cartridge RAM \\
      All other values disable access to cartridge RAM
  \end{description}
\end{register}

The RAM\_EN register is used to enable access to the cartridge SRAM if one
exists on the cartridge board. RAM access is disabled by default, but can be
enabled by writing to the \texttt{\$0000-\$1FFF} address range a value with the
bit pattern \texttt{1010} in the lower nibble. Upper bits don't matter, but any
other bit pattern in the lower nibble disables access to RAM.

When RAM access is disabled, all writes to the external RAM area
\texttt{\$A000-\$BFFF} are ignored, and reads return \texttt{\$FF}. Pan Docs
recommends disabling RAM when it's not being accessed to protect the contents
\cite{pandocs}.

\begin{speculation}
  We don't know the physical implementation of RAM\_EN, but it's certainly
  possible that the \texttt{1010} bit pattern check is done at write time and
  the register actually consists of just a single bit.
\end{speculation}

\begin{register}[H]
  \caption{\texttt{\$2000-\$3FFF} - BANK1 - MBC1 bank register 1}

  {
    \ttfamily
    \begin{tabu} to \textwidth {|X[c]|X[c]|X[c]|X[c]|X[c]|X[c]|X[c]|X[c]|}
      \everyrow{\hline}
      \hline
      U                     & U                     & U                     & W-0                                  & W-0 & W-0 & W-0 & W-1   \\
      \cellcolor{LightGray} & \cellcolor{LightGray} & \cellcolor{LightGray} & \multicolumn5{c|}{BANK1<4:0>} \\
      \rowfont{\rmfamily\small}
      bit 7                 & 6                     & 5                     & 4                                    & 3   & 2   & 1   & bit 0 \\
      \hline
    \end{tabu}
  }

  \begin{description}[leftmargin=5em, style=nextline]
    \item[bit 7-5]
      \textbf{Unimplemented}: Ignored during writes
    \item[bit 4-0]
      \textbf{BANK1<4:0>}: Bank register 1 \\
      Never contains the value \texttt{00000}. \\
      If \texttt{00000} is written, the resulting value will be \texttt{00001} instead.
  \end{description}
\end{register}

The 5-bit BANK1 register is used as the lower 5 bits of the ROM bank number
when the CPU accesses the \texttt{\$4000-\$7FFF} memory area.

MBC1 doesn't allow the BANK1 register to contain zero (bit pattern
\texttt{00000}), so the initial value at reset is \texttt{00001} and attempting
to write \texttt{00000} will write \texttt{00001} instead. This makes it
impossible to read banks \texttt{\$00}, \texttt{\$20}, \texttt{\$40} and
\texttt{\$60} from the \texttt{\$4000-\$7FFF} memory area, because those bank
numbers have \texttt{00000} in the lower bits. Due to the zero value
adjustment, requesting any of these banks actually requests the next bank (e.g.
\texttt{\$21} instead of \texttt{\$20}).

\begin{register}[H]
  \caption{\texttt{\$4000-\$5FFF} - BANK2 - MBC1 bank register 2}

  \begin{tabu} to \textwidth {|X[c]|X[c]|X[c]|X[c]|X[c]|X[c]|X[c]|X[c]|}
    \everyrow{\hline}
    \hline
    U                     & U                     & U                     & U                     & U                     & U                     & W-0                                  & W-0   \\
    \cellcolor{LightGray} & \cellcolor{LightGray} & \cellcolor{LightGray} & \cellcolor{LightGray} & \cellcolor{LightGray} & \cellcolor{LightGray} & \multicolumn2{c|}{BANK2<1:0>} \\
    \rowfont{\small}
    bit 7                 & 6                     & 5                     & 4                     & 3                     & 2                     & 1                                    & bit 0 \\
    \hline
  \end{tabu}

  \begin{description}[leftmargin=5em, style=nextline]
    \item[bit 7-2]
      \textbf{Unimplemented}: Ignored during writes
    \item[bit 1-0]
      \textbf{BANK2<1:0>}: Bank register 2
  \end{description}
\end{register}

The 2-bit BANK2 register can be used as the upper bits of the ROM bank number,
or as the 2-bit RAM bank number. Unlike BANK1, BANK2 doesn't disallow zero, so
all 2-bit values are possible.

\begin{register}
  \caption{\texttt{\$6000-\$7FFF} - MODE - MBC1 mode register}

  \begin{tabu} to \textwidth {|X[c]|X[c]|X[c]|X[c]|X[c]|X[c]|X[c]|X[c]|}
    \everyrow{\hline}
    \hline
    U                     & U                     & U                     & U                           & U & U & U & W-0   \\
    \cellcolor{LightGray} & \cellcolor{LightGray} & \cellcolor{LightGray} & \cellcolor{LightGray} & \cellcolor{LightGray} & \cellcolor{LightGray} & \cellcolor{LightGray} & MODE \\
    \rowfont{\small}
    bit 7                 & 6                     & 5                     & 4                              & 3   & 2   & 1   & bit 0 \\
    \hline
  \end{tabu}

  \begin{description}[leftmargin=5em, style=nextline]
    \item[bit 7-1]
      \textbf{Unimplemented}: Ignored during writes
    \item[bit 0]
      \textbf{MODE}: Mode register \\
      \texttt{1=} BANK2 affects accesses to \texttt{\$0000-\$3FFF}, \texttt{\$4000-\$7FFF}, \texttt{\$A000-\$BFFF} \\
      \texttt{0=} BANK2 affects only accesses to \texttt{\$4000-\$7FFF} \\
  \end{description}
\end{register}

The MODE register determines how the BANK2 register value is used during memory
accesses.

TODO.

\section{ROM banking}

In MBC1 cartridges, the A0-A13 cartridge bus signals are connected directly to
the corresponding ROM pins, and the remaining 0-7 ROM pins are controlled by
the MBC1. These remaining pins form the ROM bank number.

TODO.

\begin{table}[H]
  \caption{Mapping of physical ROM address bits in MBC1 carts}
  \ttfamily
  \begin{tabu} to \textwidth {|X[10,l]|X[2,c]|X[5,c]|X[14,c]|}
    \everyrow{\hline}
    \hline
    \rowfont{\rmfamily}
    & \multicolumn2{c|}{Bank number} & Address within bank \\
    \rowfont{\rmfamily}
    ROM address bits & 20-19 & 18-14 & 13-0 \\
    \$0000-\$3FFF, MODE = 0 & 00 & 00000 & A<13:0> \\
    \$0000-\$3FFF, MODE = 1 & BANK2 & 00000 & A<13:0> \\
    \$4000-\$7FFF & BANK2 & BANK1 & A<13:0> \\
    \hline
  \end{tabu}
\end{table}

\subsubsection{Example 1}

Let's assume we have previously written \texttt{\$18} to the BANK1 register and
\texttt{\$01} to the BANK2 register. The effective bank number during ROM reads
depends on which address range we read and on the value of the MODE register:

\begin{description}[style=nextline]
  \item[Value of BANK1 register]
  {
    \ttfamily
    \colorbox{blue!30}{10010}
  }
  \item[Value of BANK2 register]
  {
    \ttfamily
    \colorbox{red!30}{01}
  }
  \item[Effective ROM bank number (reading \$4000-\$7FFF)]
  {
    \ttfamily
    \colorbox{red!30}{01}\colorbox{blue!30}{10010} (68 = \$44)
  }
  \item[Effective ROM bank number (reading \$0000-\$3FFF, MODE = 0)]
  {
    \ttfamily
    \colorbox{gray!10}{00}\colorbox{gray!10}{00000} (0 = \$00)
  }
  \item[Effective ROM bank number (reading \$0000-\$3FFF, MODE = 1)]
  {
    \ttfamily
    \colorbox{red!30}{01}\colorbox{gray!10}{00000} (64 = \$40)
  }
\end{description}

\subsubsection{Example 2}

Let's assume we have previously requested ROM bank number \texttt{68}, MBC1
mode is \texttt{0}, and we are now reading a byte from \texttt{\$72A7}. The
actual physical ROM address that will be read is going to be \texttt{\$1132A7}
and is constructed in the following way:

\begin{description}[leftmargin=15em,style=nextline]
  \item[Value of BANK1 register]
  {
    \ttfamily
    \colorbox{blue!30}{00100}
  }
  \item[Value of BANK2 register]
  {
    \ttfamily
    \colorbox{red!30}{10}
  }
  \item[ROM bank number]
  {
    \ttfamily
    \colorbox{red!30}{10}\colorbox{blue!30}{00100} (68 = \$44)
  }
  \item[Address being read]
  {
    \ttfamily
    \colorbox{gray!10}{01}\colorbox{green!30}{11 0010 1010 0111} (\$72A7)
  }
  \item[Actual physical ROM address]
  {
    \ttfamily
    \colorbox{red!30}{1 0}\colorbox{blue!30}{001 00}\colorbox{green!30}{11 0010 1010 0111} (\$1132A7)
  }
\end{description}

\section{ROM banking (multicarts)}

TODO.

\section{RAM banking}

TODO.

\section{RAM banking (multicarts)}

TODO.

\section{Developing a ROM that uses MBC1}

TODO.

\section{Dumping MBC1 carts}

TODO.

\section{Emulating MBC1}

TODO.
