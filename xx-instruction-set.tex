\chapter{Sharp LR35902 instruction set}

\section{8-bit load and store instructions}

\section{16-bit load and store instructions}

\section{8-bit arithmetic instructions}

\section{16-bit arithmetic instructions}

\section{Rotate, shift, and bit operation instructions}

\section{Control flow instructions}

\subsection{JP nn}
\label{inst:JP}

Unconditional jump to the absolute address specified by the operand \texttt{nn}.

\begin{description}[leftmargin=9em, style=nextline]
  \item[Opcode + data]
    \hex{C3}, LSB of \texttt{nn}, MSB of \texttt{nn}
  \item[Length]
    3 bytes
  \item[Duration]
    4 machine cycles
  \item[Timing] \parbox{0.8\textwidth}{
    \begin{tikztimingtable}[timing/wscale=0.8]
      Purpose & X 8D{Decode}   8D{LSB of nn}  8D{MSB of nn}  8D{Internal delay} ; [opacity=0.4] 8D{Decode}   X \\
      Memory  & X 8D{Read: PC} 8D{Read: PC+1} 8D{Read: PC+2} 8U                 ; [opacity=0.4] 8D{Read: nn} X \\
    \end{tikztimingtable}}
\item[Pseudocode] \begin{verbatim}
opcode = read(PC++)
if opcode == 0xC3:
  nn = unsigned_16(lsb=read(PC++), msb=read(PC++))
  PC = nn
\end{verbatim}
\end{description}

\subsection{JP HL}
\label{inst:JP_HL}

Unconditional jump to the absolute address specified by the register HL.

\begin{description}[leftmargin=9em, style=nextline]
  \item[Opcode + data]
    \hex{E9}
  \item[Length]
    1 bytes
  \item[Duration]
    1 machine cycle
  \item[Timing] \parbox{0.8\textwidth}{
    \begin{tikztimingtable}[timing/wscale=0.8]
      Purpose & X 8D{Decode}   ; [opacity=0.4] 8D{Decode}   X \\
      Memory  & X 8D{Read: PC} ; [opacity=0.4] 8D{Read: HL} X \\
    \end{tikztimingtable}}
\item[Pseudocode] \begin{verbatim}
opcode = read(PC++)
if opcode == 0xE9:
  PC = HL
\end{verbatim}
\end{description}

\begin{warning}
  In some documentation this instruction is written as \texttt{JP [HL]}. This
  is very misleading, since brackets are usually used to indicate a memory
  read, and this instruction simply copies the value of HL to PC.
\end{warning}

\subsection{JP cc, nn}
\label{inst:JP_cc}

Conditional jump to the absolute address specified by the operand \texttt{nn}, depending on the condition \texttt{cc}.

Note that the target address is read even when the condition is false!

\begin{description}[leftmargin=9em, style=nextline]
  \item[Opcode + data]
    \hex{C2}/\hex{D2}/\hex{CA}/\hex{DA}, LSB of \texttt{nn}, MSB of \texttt{nn}
  \item[Length]
    3 bytes
  \item[Duration]
    3 machine cycles (cc=false), or 4 machine cycles (cc=true)
  \item[Timing (cc=false)] \parbox{0.8\textwidth}{
    \begin{tikztimingtable}[timing/wscale=0.8]
      Purpose & X 8D{Decode}   8D{LSB of nn}  8D{MSB of nn}  ; [opacity=0.4] 8D{Decode}     X \\
      Memory  & X 8D{Read: PC} 8D{Read: PC+1} 8D{Read: PC+2} ; [opacity=0.4] 8D{Read: PC+3} X \\
    \end{tikztimingtable}}
  \item[Timing (cc=true)] \parbox{0.8\textwidth}{
    \begin{tikztimingtable}[timing/wscale=0.8]
      Purpose & X 8D{Decode}   8D{LSB of nn}  8D{MSB of nn}  8D{Internal delay} ; [opacity=0.4] 8D{Decode}   X \\
      Memory  & X 8D{Read: PC} 8D{Read: PC+1} 8D{Read: PC+2} 8U                 ; [opacity=0.4] 8D{Read: nn} X \\
    \end{tikztimingtable}}
\item[Pseudocode] \begin{verbatim}
opcode = read(PC++)
if opcode in [0xC2, 0xD2, 0xCA, 0xDA]:
  nn = unsigned_16(lsb=read(PC++), msb=read(PC++))
  if F.check_condition(cc):
    PC = nn
\end{verbatim}
\end{description}

\subsection{JR r}
\label{inst:JR}

Unconditional jump to the relative address specified by the signed operand \texttt{r}.

\begin{description}[leftmargin=9em, style=nextline]
  \item[Opcode + data]
    \hex{18}, offset \texttt{r}
  \item[Length]
    2 bytes
  \item[Duration]
    3 machine cycles
  \item[Timing] \parbox{0.8\textwidth}{
    \begin{tikztimingtable}[timing/wscale=0.8]
      Purpose & X 8D{Decode}   8D{Value of r} 8D{Internal delay} ; [opacity=0.4] 8D{Decode}       X \\
      Memory  & X 8D{Read: PC} 8D{Read: PC+1} 8U                 ; [opacity=0.4] 8D{Read: PC+2+r} X \\
    \end{tikztimingtable}}
\item[Pseudocode] \begin{verbatim}
opcode = read(PC++)
if opcode == 0x18:
  r = signed_8(read(PC++))
  PC = PC + r
\end{verbatim}
\end{description}

\subsection{JR cc, r}
\label{inst:JR_cc}

Conditional jump to the relative address specified by the signed operand \texttt{r}, depending on the condition \texttt{cc}.

\begin{description}[leftmargin=9em, style=nextline]
  \item[Opcode + data]
    \hex{20}/\hex{30}/\hex{28}/\hex{38}, offset \texttt{r}
  \item[Length]
    2 bytes
  \item[Duration]
    2 machine cycles (cc=false), or 3 machine cycles (cc=true)
  \item[Timing (cc=false)] \parbox{0.8\textwidth}{
    \begin{tikztimingtable}[timing/wscale=0.8]
      Purpose & X 8D{Decode}   8D{Value of r} ; [opacity=0.4] 8D{Decode}     X \\
      Memory  & X 8D{Read: PC} 8D{Read: PC+1} ; [opacity=0.4] 8D{Read: PC+2} X \\
    \end{tikztimingtable}}
  \item[Timing (cc=true)] \parbox{0.8\textwidth}{
    \begin{tikztimingtable}[timing/wscale=0.8]
      Purpose & X 8D{Decode}   8D{Value of r} 8D{Internal delay} ; [opacity=0.4] 8D{Decode}       X \\
      Memory  & X 8D{Read: PC} 8D{Read: PC+1} 8U                 ; [opacity=0.4] 8D{Read: PC+2+r} X \\
    \end{tikztimingtable}}
\item[Pseudocode] \begin{verbatim}
opcode = read(PC++)
if opcode in [0x20, 0x30, 0x28, 0x38]:
  r = signed_8(read(PC++))
  if F.check_condition(cc):
    PC = PC + r
\end{verbatim}
\end{description}

\subsection{CALL nn}
\label{inst:CALL}

Unconditional function call to the absolute address specified by the operand \texttt{nn}.

\begin{description}[leftmargin=9em, style=nextline]
  \item[Opcode + data]
    \hex{CD}, LSB of \texttt{nn}, MSB of \texttt{nn}
  \item[Length]
    3 bytes
  \item[Duration]
    6 machine cycles
  \item[Timing] \parbox{0.8\textwidth}{
    \begin{tikztimingtable}[timing/wscale=0.8]
      Purpose & X 8D{Decode}   8D{LSB of nn}  8D{MSB of nn}  8D{Internal delay} 8D{MSB of PC+3} 8D{LSB of PC+3} ; [opacity=0.4] 8D{Decode}   X \\
      Memory  & X 8D{Read: PC} 8D{Read: PC+1} 8D{Read: PC+2} 8U                 8D{Write: SP-1} 8D{Write: SP-2} ; [opacity=0.4] 8D{Read: nn} X \\
    \end{tikztimingtable}}
\item[Pseudocode] \begin{verbatim}
opcode = read(PC++)
if opcode == 0xCD:
  nn = unsigned_16(lsb=read(PC++), msb=read(PC++))
  write(--SP, msb(PC))
  write(--SP, lsb(PC))
  PC = nn
\end{verbatim}
\end{description}

\subsection{CALL cc, nn}
\label{inst:CALL_cc}

Conditional function call to the absolute address specified by the operand \texttt{nn}, depending on the condition \texttt{cc}.

\begin{description}[leftmargin=9em, style=nextline]
  \item[Opcode + data]
    \hex{C4}/\hex{D4}/\hex{CC}/\hex{DC}, LSB of \texttt{nn}, MSB of \texttt{nn}
  \item[Length]
    3 bytes
  \item[Duration]
    3 machine cycles (cc=false), or 6 machine cycles (cc=true)
  \item[Timing (cc=false)] \parbox{0.8\textwidth}{
    \begin{tikztimingtable}[timing/wscale=0.8]
      Purpose & X 8D{Decode}   8D{LSB of nn}  8D{MSB of nn}  ; [opacity=0.4] 8D{Decode}     X \\
      Memory  & X 8D{Read: PC} 8D{Read: PC+1} 8D{Read: PC+2} ; [opacity=0.4] 8D{Read: PC+3} X \\
    \end{tikztimingtable}}
  \item[Timing (cc=true)] \parbox{0.8\textwidth}{
    \begin{tikztimingtable}[timing/wscale=0.8]
      Purpose & X 8D{Decode}   8D{LSB of nn}  8D{MSB of nn}  8D{Internal delay} 8D{MSB of PC+3} 8D{LSB of PC+3} ; [opacity=0.4] 8D{Decode}   X \\
      Memory  & X 8D{Read: PC} 8D{Read: PC+1} 8D{Read: PC+2} 8U                 8D{Write: SP-1} 8D{Write: SP-2} ; [opacity=0.4] 8D{Read: nn} X \\
    \end{tikztimingtable}}
\item[Pseudocode] \begin{verbatim}
opcode = read(PC++)
if opcode in [0xC4, 0xD4, 0xCC, 0xDC]:
  nn = unsigned_16(lsb=read(PC++), msb=read(PC++))
  if F.check_condition(cc):
    write(--SP, msb(PC))
    write(--SP, lsb(PC))
    PC = nn
\end{verbatim}
\end{description}

\subsection{RET}
\label{inst:RET}

Unconditional return from function.

\begin{description}[leftmargin=9em, style=nextline]
  \item[Opcode + data]
    \hex{C9}
  \item[Length]
    1 byte
  \item[Duration]
    4 machine cycles
  \item[Timing] \parbox{0.8\textwidth}{
    \begin{tikztimingtable}[timing/wscale=0.8]
      Purpose & X 8D{Decode}   8D{LSB of PC} 8D{MSB of PC}  8D{Internal delay} ; [opacity=0.4] 8D{Decode}       X \\
      Memory  & X 8D{Read: PC} 8D{Read: SP}  8D{Read: SP+1} 8U                 ; [opacity=0.4] 8D{Read: new PC} X \\
    \end{tikztimingtable}}
\item[Pseudocode] \begin{verbatim}
opcode = read(PC++)
if opcode == 0xC9:
  PC = unsigned_16(lsb=read(SP++), msb=read(SP++))
\end{verbatim}
\end{description}

\subsection{RET cc}
\label{inst:RET_cc}

Conditional return from function, depending on the condition \texttt{cc}.

\begin{description}[leftmargin=9em, style=nextline]
  \item[Opcode + data]
    \hex{C0}/\hex{D0}/\hex{C8}/\hex{D8}
  \item[Length]
    1 byte
  \item[Duration]
    2 machine cycles (cc=false), or 5 machine cycles (cc=true)
  \item[Timing (cc=false)] \parbox{0.8\textwidth}{
    \begin{tikztimingtable}[timing/wscale=0.8]
      Purpose & X 8D{Decode}   8D{Internal delay} ; [opacity=0.4] 8D{Decode}     X \\
      Memory  & X 8D{Read: PC} 8U                 ; [opacity=0.4] 8D{Read: PC+1} X \\
    \end{tikztimingtable}}
  \item[Timing (cc=true)] \parbox{0.8\textwidth}{
    \begin{tikztimingtable}[timing/wscale=0.8]
      Purpose & X 8D{Decode}   8D{Internal delay} 8D{LSB of PC} 8D{MSB of PC}  8D{Internal delay} ; [opacity=0.4] 8D{Decode}       X \\
      Memory  & X 8D{Read: PC} 8U                 8D{Read: SP}  8D{Read: SP+1} 8U                 ; [opacity=0.4] 8D{Read: new PC} X \\
    \end{tikztimingtable}}
\item[Pseudocode] \begin{verbatim}
opcode = read(PC++)
if opcode in [0xC0, 0xD0, 0xC8, 0xD8]:
  if F.check_condition(cc):
    PC = unsigned_16(lsb=read(SP++), msb=read(SP++))
\end{verbatim}
\end{description}

\subsection{RETI}
\label{inst:RETI}

Unconditional return from function. Also enables interrupts by setting IME=1.

\begin{description}[leftmargin=9em, style=nextline]
  \item[Opcode + data]
    \hex{D9}
  \item[Length]
    1 byte
  \item[Duration]
    4 machine cycles
  \item[Timing] \parbox{0.8\textwidth}{
    \begin{tikztimingtable}[timing/wscale=0.8]
      Purpose & X 8D{Decode}   8D{LSB of PC} 8D{MSB of PC}  8D{Internal delay} ; [opacity=0.4] 8D{Decode}       X \\
      Memory  & X 8D{Read: PC} 8D{Read: SP}  8D{Read: SP+1} 8U                 ; [opacity=0.4] 8D{Read: new PC} X \\
    \end{tikztimingtable}}
\item[Pseudocode] \begin{verbatim}
opcode = read(PC++)
if opcode == 0xD9:
  PC = unsigned_16(lsb=read(SP++), msb=read(SP++))
  IME = 1
\end{verbatim}
\end{description}

\subsection{RST n}
\label{inst:RST}

Unconditional function call to the absolute fixed address defined by the opcode.

\begin{description}[leftmargin=9em, style=nextline]
  \item[Opcode + data]
    \hex{C7}/\hex{D7}/\hex{E7}/\hex{F7}/\hex{CF}/\hex{DF}/\hex{EF}/\hex{FF}
  \item[Length]
    1 byte
  \item[Duration]
    4 machine cycles
  \item[Timing] \parbox{0.8\textwidth}{
    \begin{tikztimingtable}[timing/wscale=0.8]
      Purpose & X 8D{Decode}   8D{Internal delay} 8D{MSB of PC+1} 8D{LSB of PC+1} ; [opacity=0.4] 8D{Decode}       X \\
      Memory  & X 8D{Read: PC} 8U                 8D{Write: SP-1} 8D{Write: SP-2} ; [opacity=0.4] 8D{Read: new PC} X \\
    \end{tikztimingtable}}
\item[Pseudocode] \begin{verbatim}
opcode = read(PC++)
if opcode in [0xC7, 0xD7, 0xE7, 0xF7, 0xCF, 0xDF, 0xEF, 0xFF]:
  n = rst_address(opcode)
  write(--SP, msb(PC))
  write(--SP, lsb(PC))
  PC = unsigned_16(lsb=n, msb=0x00)
\end{verbatim}
\end{description}

\section{Miscellaneous instructions}

\subsection{HALT}
\label{inst:HALT}

\subsection{STOP}
\label{inst:STOP}

\subsection{DI}
\label{inst:DI}

\subsection{EI}
\label{inst:EI}

\subsection{CCF}
\label{inst:CCF}

\subsection{SCF}
\label{inst:SCF}

\subsection{NOP}
\label{inst:NOP}

No-operation. This instruction doesn't do anything, but can be used to add a
delay of one machine cycle and increment PC by one.

\begin{description}[leftmargin=9em, style=nextline]
  \item[Opcode]
    \hex{00}
  \item[Length]
    1 byte
  \item[Duration]
    1 machine cycle
  \item[Timing] \parbox{0.8\textwidth}{
    \begin{tikztimingtable}[timing/wscale=0.8]
      Purpose & X 8D{Decode}   ; [opacity=0.4] 8D{Decode}     X \\
      Memory  & X 8D{Read: PC} ; [opacity=0.4] 8D{Read: PC+1} X \\
    \end{tikztimingtable}}
\item[Pseudocode] \begin{verbatim}
opcode = read(PC++)
if opcode == 0x00:
  // nothing
\end{verbatim}
\end{description}

\subsection{DAA}
\label{inst:DAA}

\subsection{CPL}
\label{inst:CPL}
