%!TEX root = ../../gbctr.tex
%!TEX program = lualatex
\providecommand{\main}{../..}
\documentclass[\main/gbctr.tex]{subfiles}
\begin{document}

\chapter{CPU core timing}

\section{Fetch/execute overlap}

Sharp SM83 uses a microprocessor design technique known as \emph{fetch/execute
overlap} to improve CPU performance by doing opcode fetches in parallel with
instruction execution whenever possible. Since the CPU can only perform one
memory access per M-cycle, it is worth it to try to do memory operations as
soon as possible. Also, when doing a memory read, the CPU cannot use the data
during the same M-cycle so the true minimum effective duration of instructions
is 2 machine cycles, not 1 machine cycle.

Every instruction needs one machine cycle for the fetch stage, and at least one
machine cycle for the decode/execute stage. However, the fetch stage of an
instruction always overlaps with the last machine cycle of the execute stage of
the previous instruction. The overlapping execute stage cycle may still do some
work (e.g. ALU operation and/or register writeback) but memory access is
reserved for the fetch stage of the next instruction.

Since all instructions effectively last one machine cycle longer, fetch/execute
overlap is usually ignored in documentation intended for programmers. It is
much easier to think of a program as a sequence of non-overlapping instructions
and consider only the execute stages when calculating instruction durations.
However, when emulating a SM83 CPU core, understanding and emulating the
overlap can be useful.

\begin{warning}
  Sharp SM831x is a family of single-chip SoCs from Sharp that use the SM83 CPU
  core, and their datasheet \cite{sm831x} includes a description of
  fetch/execute overlap. However, the description is not completely correct and
  can in fact be misleading.

  For example, the timing diagram includes an instruction that does not involve
  opcode fetch at all, and memory operations for two instructions are shown to
  happen at the same time, which is not possible.
\end{warning}

\subsection{Fetch/execute overlap timing example}

Let's assume the CPU is executing a program that starts from the address
\hex{1000} and contains the following instructions:\\
{
  \ttfamily
  \hex{1000}:\colorbox{blue!30}{INC A}
}\\
{
  \ttfamily
  \hex{1001}:\colorbox{yellow!30}{LDH (n), A}
}\\
{
  \ttfamily
  \hex{1003}:\colorbox{green!30}{RST \hex{08}}
}\\
{
  \ttfamily
  \hex{0008}:\colorbox{red!30}{NOP}
}

The following timing diagram shows all memory operations done by the CPU, and
the fetch and execute stages of each instruction:

\begin{figure}[H]
  \centering
  \begin{tikztimingtable}[timing/wscale=0.6]
    CLK 4MHz & L 10{8{C}} C \\
    PHI 1MHz & L 20{4C} C \\
    Mem R/W & X ; [fill=blue!30] 8D{R: opcode} ; [fill=yellow!30] 8D{R: opcode} 8D{R: n} 8D{W: A} ; [fill=green!30] 8D{R: opcode} ; 8X ; [fill=green!30] 8D{W: msb(PC)} 8D{W: lsb(PC)} ; [fill=red!30] 8D{R: opcode} ; [opacity=0.4] 8D{R: opcode} X \\
    Before \texttt{INC A} & ; [opacity=0.4] 9D{execute} ; 73X \\
    \texttt{INC A} & X ; [fill=blue!30] 8D{M1: fetch} 8D{M2: execute} ; 65X \\
    \texttt{LDH (n), A} & X 8X ; [fill=yellow!30] 8D{M1: fetch} 24D{M2-4: execute} ; 41X \\
    \texttt{RST \hex{08}} & X 32X ; [fill=green!30] 8D{M1: fetch} 32D{M2-5: execute} ; 9X \\
    \texttt{NOP} & X 64X ; [fill=red!30] 8D{M1: fetch} 8D{M2: execute} ; X \\
    After \texttt{NOP} & X 72X ; [opacity=0.4] 9D{M1: fetch} \\
  \end{tikztimingtable}
\caption{Fetch/execute overlap example}
\end{figure}

\end{document}
