%!TEX root = ../../gbctr.tex
%!TEX program = xelatex
\providecommand{\main}{../..}
\documentclass[\main/gbctr.tex]{subfiles}
\begin{document}

\chapter{MBC2 mapper chip}

MBC2 supports ROM sizes up to 2 Mbit (16 banks of \hex{4000} bytes) and
includes an internal 512x4 bit RAM array, which is its unique feature. The
information in this section is based on my MBC2 research, Tauwasser's research
notes \cite{tauwasser_mbc2}, and Pan Docs \cite{pandocs}.

\begin{speculation}
  MBC1 is strictly more powerful than MBC2 because it supports more ROM and
  RAM. This raises a very important question: why does MBC2 exist? It's
  possible that Nintendo tried to integrate a small amount of RAM on the MBC
  chip for cost reasons, but it seems that this didn't work out very well since
  all later MBCs revert this design decision and use separate RAM chips.
\end{speculation}

\section{MBC2 registers}

\begin{caveat}
  These registers don't have any standard names and are usually referred to
  using one of their addresses or purposes instead. This document uses names to
  clarify which register is meant when referring to one.
\end{caveat}

The MBC2 chip includes two registers that affect the behaviour of the chip. The
registers are mapped a bit differently compared to other MBCs. Both registers
are accessible within \hexrange{0000}{3FFF}, and within that range, the
register is chosen based on the A8 address signal. In practice, this means that
the registers are mapped to memory in an alternating pattern. For example,
\hex{0000}, \hex{2000} and \hex{3000} are RAMG, and \hex{0100}, \hex{2100} and
\hex{3100} are ROMB. Both registers are smaller than 8 bits, and unused bits
are simply ignored during writes. The registers are not directly readable.

\begin{register}[H]
  \caption{\hexrange{0000}{3FFF} when A8=\bin{0} - RAMG - MBC2 RAM gate register}
  {
    \ttfamily
    \begin{tabularx}{\linewidth}{|X|X|X|X|X|X|X|X|}
      \hline
      U & U & U & U & W-0 & W-0 & W-0 & W-0 \\
      \hline
      \cellcolor{LightGray} & \cellcolor{LightGray} & \cellcolor{LightGray} & \cellcolor{LightGray} & \multicolumn{4}{c|}{RAMG<3:0>} \\
      \hline
      bit 7 & 6 & 5 & 4 & 3 & 2 & 1 & bit 0 \\
      \hline
    \end{tabularx}{\parfillskip=0pt\par}
  }

  \begin{description}[leftmargin=5em, style=nextline]
    \item[bit 7-4]
      \textbf{Unimplemented}: Ignored during writes
    \item[bit 3-0]
      \textbf{RAMG<3:0>}: RAM gate register\\
      \bin{1010}= enable access to chip RAM\\
      All other values disable access to chip RAM
  \end{description}
\end{register}

The 4-bit MBC2 RAMG register works in a similar manner as MBC1 RAMG, so the
upper bits don't matter and only the bit pattern \bin{1010} enables access to
RAM.

When RAM access is disabled, all writes to the external RAM area
\hexrange{A000}{BFFF} are ignored, and reads return undefined values. Pan Docs
recommends disabling RAM when it's not being accessed to protect the contents
\cite{pandocs}.

\begin{speculation}
  We don't know the physical implementation of RAMG, but it's certainly
  possible that the \bin{1010} bit pattern check is done at write time and the
  register actually consists of just a single bit.
\end{speculation}

\begin{register}[H]
  \caption{\hexrange{0000}{3FFF} when A8=\bin{1} - ROMB - MBC2 ROM bank register}
  {
    \ttfamily
    \begin{tabularx}{\linewidth}{|X|X|X|X|X|X|X|X|}
      \hline
      U & U & U & U & W-0 & W-0 & W-0 & W-1 \\
      \hline
      \cellcolor{LightGray} & \cellcolor{LightGray} & \cellcolor{LightGray} & \cellcolor{LightGray} & \multicolumn{4}{c|}{ROMB<3:0>} \\
      \hline
      bit 7 & 6 & 5 & 4 & 3 & 2 & 1 & bit 0 \\
      \hline
    \end{tabularx}{\parfillskip=0pt\par}
  }

  \begin{description}[leftmargin=5em, style=nextline]
    \item[bit 7-4]
      \textbf{Unimplemented}: Ignored during writes
    \item[bit 3-0]
      \textbf{ROMB<3:0>}: ROM bank register\\
      Never contains the value \bin{0000}.\\
      If \bin{0000} is written, the resulting value will be \bin{0001} instead.
  \end{description}
\end{register}

The 4-bit ROMB register is used as the ROM bank number when the CPU accesses
the \hexrange{4000}{7FFF} memory area.

Like MBC1 BANK1, the MBC2 ROMB register doesn't allow zero (bit pattern
\bin{0000}) in the register, so any attempt to write \bin{0000} writes
\bin{0001} instead.

\section{ROM in the \hexrange{0000}{7FFF} area}

In MBC2 cartridges, the A0-A13 cartridge bus signals are connected directly to
the corresponding ROM pins, and the remaining ROM pins (A14-A17) are controlled
by the MBC2. These remaining pins form the ROM bank number.

When the \hexrange{0000}{3FFF} address range is accessed, the effective bank
number is always 0.

When the \hexrange{4000}{7FFF} address range is accessed, the effective bank
number is the current ROMB register value.

\begin{table}[H]
  \caption{Mapping of physical ROM address bits in MBC2 carts}
  \centering
  \begin{tabular}{|l|c|c|}
    \hline
    & \multicolumn{2}{c|}{ROM address bits} \\
    Accessed address & Bank number & Address within bank \\
    \hline
    & 17-14 & 13-0 \\
    \hline
    \hexrange{0000}{3FFF} & \bin{0000} & A<13:0> \\
    \hline
    \hexrange{4000}{7FFF} & ROMB & A<13:0> \\
    \hline
  \end{tabular}
\end{table}

\section{RAM in the \hexrange{A000}{BFFF} area}

All MBC2 carts include SRAM, because it is located directly inside the MBC2
chip. These cartridges never use a separate RAM chip, but battery backup
circuitry and a battery are optional. If RAM is not enabled with the RAMG
register, all reads return undefined values and writes have no effect.

MBC2 RAM is only 4-bit RAM, so the upper 4 bits of data do not physically exist
in the chip. When writing to it, the upper 4 bits are ignored. When reading
from it, the upper 4 data signals are not driven by the chip, so their content
is undefined and should not be relied on.

MBC2 RAM consists of 512 addresses, so only A0-A8 matter when accessing the RAM
region. There is no banking, and the \hexrange{A000}{BFFF} area is larger than
the RAM, so the addresses wrap around. For example, accessing \hex{A000} is the
same as accessing \hex{A200}, so it is possible to write to the former address
and later read the written data using the latter address.

\begin{table}[H]
  \caption{Mapping of physical RAM address bits in MBC2 carts}
  \centering
  \begin{tabular}{|l|c|}
    \hline
    & RAM address bits \\
    Accessed address & \\
    \hline
    & 8-0 \\
    \hline
    \hexrange{A000}{BFFF} & A<8:0> \\
    \hline
  \end{tabular}
\end{table}

\section{Dumping MBC2 carts}

MBC2 cartridges are very simple to dump. The total number of banks is read from
the header, and each bank is read one byte at a time. ROMB zero adjustment must
be considered in the ROM dumping code, but this only means that bank 0 should
be read from \hexrange{0000}{3FFF} and not from \hexrange{4000}{7FFF} like
other banks.

\begin{listing}[H]
  \inputminted[frame=lines]{python}{code-snippets/mbc2_rom_dump.py}
  \caption{Python pseudo-code for MBC2 ROM dumping}
\end{listing}

\end{document}
