%!TEX root = ../../gbctr.tex
%!TEX program = xelatex
\providecommand{\main}{../..}
\documentclass[\main/gbctr.tex]{subfiles}
\begin{document}

\chapter{Boot ROM}

\begin{register}[H]
  \caption{\hex{FF50} - BOOT - Boot ROM lock register}
  {
    \ttfamily
    \begin{tabularx}{\linewidth}{|X|X|X|X|X|X|X|X|}
      \hline
      U-1                     & U-1                     & U-1                     & U-1                     & U-1                     & U-1                     & U-1                     & R/W-0     \\
      \hline
      \cellcolor{LightGray} - & \cellcolor{LightGray} - & \cellcolor{LightGray} - & \cellcolor{LightGray} - & \cellcolor{LightGray} - & \cellcolor{LightGray} - & \cellcolor{LightGray} - & BOOT\_OFF \\
      \hline
      bit 7                   & 6                       & 5                       & 4                       & 3                       & 2                       & 1                       & bit 0     \\
      \hline
    \end{tabularx}{\parfillskip=0pt\par}
  }

  \begin{description}[leftmargin=5em, style=nextline]
    \item[bit 7-1]
      \textbf{Unimplemented}: Read as \bit{1}
    \item[bit 0]
      \textbf{BOOT\_OFF}: Boot ROM lock bit\\
      \bin{1}= Boot ROM is disabled and \hexrange{0000}{00FF} works normally.\\
      \bin{0}= Boot ROM is active and intercepts accesses to \hexrange{0000}{00FF}.

      \bigskip

      BOOT\_OFF can only transition from \bin{0} to \bin{1}, so once \bin{1}
      has been written, the boot ROM is permanently disabled until the next
      system reset. Writing \bin{0} when BOOT\_OFF is \bin{0} has no effect and
      doesn't lock the boot ROM.
  \end{description}
\end{register}

\end{document}
