%!TEX root = ../../gbctr.tex
%!TEX program = xelatex
\providecommand{\main}{../..}
\documentclass[\main/gbctr.tex]{subfiles}
\begin{document}

\chapter{Boot ROM}

The Game Boy SoC includes a small embedded boot ROM, which can be mapped to the
\hexrange{0000}{00FF} memory area. While mapped, all reads from this area are
handled by the boot ROM instead of the external cartridge, and all writes to
this area are ignored and cannot be seen by external hardware (e.g. the
cartridge MBC).

The boot ROM is enabled by default, so when the system exits the reset state
and the CPU starts execution from address \hex{0000}, it executes the boot ROM
instead of instructions from the cartridge ROM. The boot ROM is responsible for
showing the initial logo, and checking that a valid cartridge is inserted into
the system. If the cartridge is valid, the boot ROM unmaps itself before
execution of the cartridge ROM starts at \hex{0100}. The cartridge ROM has no
chance of executing any instructions before the boot ROM is unmapped, which
prevents the boot ROM from being read byte by byte in normal conditions.

\begin{warning}
  Don't confuse the boot ROM with the additional SNES ROM in SGB/SGB2 that is
  executed by the SNES CPU.
\end{warning}

\begin{register}[H]
  \caption{\hex{FF50} - BOOT - Boot ROM lock register}
  {
    \ttfamily
    \begin{tabularx}{\linewidth}{|C|C|C|C|C|C|C|C|}
      \hline
      U-1 & U-1 & U-1 & U-1 & U-1 & U-1 & U-1 & R/W-0 \\
      \hline
      \cellcolor{LightGray} - & \cellcolor{LightGray} - & \cellcolor{LightGray} - & \cellcolor{LightGray} - & \cellcolor{LightGray} - & \cellcolor{LightGray} - & \cellcolor{LightGray} - & BOOT\_OFF \\
      \hline
      bit 7 & 6 & 5 & 4 & 3 & 2 & 1 & bit 0 \\
      \hline
    \end{tabularx}{\parfillskip=0pt\par}
  }

  \begin{description}[leftmargin=5em, style=nextline]
    \item[bit 7-1]
      \textbf{Unimplemented}: Read as \bit{1}
    \item[bit 0]
      \textbf{BOOT\_OFF}: Boot ROM lock bit\\
      \bin{1}= Boot ROM is disabled and \hexrange{0000}{00FF} works normally.\\
      \bin{0}= Boot ROM is active and intercepts accesses to \hexrange{0000}{00FF}.

      \bigskip

      BOOT\_OFF can only transition from \bin{0} to \bin{1}, so once \bin{1}
      has been written, the boot ROM is permanently disabled until the next
      system reset. Writing \bin{0} when BOOT\_OFF is \bin{0} has no effect and
      doesn't lock the boot ROM.
  \end{description}
\end{register}

The 1-bit BOOT register controls mapping of the boot ROM. Once \bit{1} has been
written to it to unmap the boot ROM, it can only be mapped again by resetting
the system.

\section{Boot ROM types}

\begin{table}[H]
  \caption{Summary of boot ROM file hashes}
  \centering
  \small
  \begin{tabular}{|l|l|l|l|}
    \hline
    Type & CRC32 & MD5 & SHA1 \\
    \hline
    DMG & \texttt{59c8598e} & \texttt{32fbbd84168d3482956eb3c5051637f5} & \texttt{4ed31ec6b0b175bb109c0eb5fd3d193da823339f} \\
    \hline
    MGB & \texttt{e6920754} & \texttt{71a378e71ff30b2d8a1f02bf5c7896aa} & \texttt{4e68f9da03c310e84c523654b9026e51f26ce7f0} \\
    \hline
    SGB & \texttt{ec8a83b9} & \texttt{d574d4f9c12f305074798f54c091a8b4} & \texttt{aa2f50a77dfb4823da96ba99309085a3c6278515} \\
    \hline
    SGB2 & \texttt{53d0dd63} & \texttt{e0430bca9925fb9882148fd2dc2418c1} & \texttt{93407ea10d2f30ab96a314d8eca44fe160aea734} \\
    \hline
    DMG0 & \texttt{c2f5cc97} & \texttt{a8f84a0ac44da5d3f0ee19f9cea80a8c} & \texttt{8bd501e31921e9601788316dbd3ce9833a97bcbc} \\
    \hline
  \end{tabular}
\end{table}

\subsection{DMG boot ROM}

The most common boot ROM is the DMG boot ROM used in almost all original Game
Boy units. If a valid cartridge is inserted, the boot ROM scrolls a logo to the
center of the screen, and plays a "di-ding" sound recognizable by most people
who have used Game Boy consoles.

This boot ROM was originally dumped by neviksti in 2003 by decapping the Game
Boy SoC and visually inspecting every single bit.

\subsection{MGB boot ROM}

This boot ROM was originally dumped by Bennvenn in 2014 by using a simple clock
glitching method that only requires one wire.

\subsection{SGB boot ROM}

This boot ROM was originally dumped by Costis Sideris in 2009 by using an
FPGA-based clock glitching method \cite{costis_sgb}.

\subsection{SGB2 boot ROM}

This boot ROM was originally dumped by gekkio in 2015 by using a Teensy 3.1
-based clock glitching method \cite{gekkio_sgb2}.

\subsection{Early DMG boot ROM}

Very early original Game Boy units released in Japan (often called "DMG0")
included the launch version "DMG-CPU" SoC chip, which used a different boot ROM
than later units.

This boot ROM was originally dumped by gekkio in 2016 by using a clock
glitching method invented by BennVenn.

\end{document}
