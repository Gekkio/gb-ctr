%!TEX root = ../../gbctr.tex
%!TEX program = lualatex
\providecommand{\main}{../..}
\documentclass[\main/gbctr.tex]{subfiles}
\begin{document}

\chapter{DMA (Direct Memory Access)}

\section{Object Attribute Memory (OAM) DMA}

OAM DMA is a high-throughput mechanism for copying data to the OAM area (a.k.a.
Object Attribute Memory, a.k.a. sprite memory). It can copy one byte per
machine cycle without involving the CPU at all, which is much faster than the
fastest possible \texttt{memcpy} routine that can be written with the SM83
instruction set. However, a transfer cannot be cancelled and the transfer
length cannot be controlled, so the DMA transfer always updates the entire OAM
area (= 160 bytes) even if you actually want to just update the first couple of
bytes.

The Game Boy CPU chip contains a DMA controller that coordinates transfers
between a \emph{source area} and the \emph{OAM area} independently of the CPU.
While a transfer is in progress, it takes control of the source bus and the OAM
area, so some precaution is needed with memory accesses (including instruction
fetches) to avoid OAM DMA bus conflicts. OAM DMA uses a different address
decoding scheme than normal memory accesses, so the source bus is always either
the external bus or the video RAM bus, and the contents normally visible to the
CPU in the \hexrange{FE00}{FFFF} address range cannot be used as a source for
OAM DMA transfers.

The upper 8 bits of the OAM DMA source address are stored in the DMA register,
while the lower 8 bits used by both the source and target address are stored in
the DMA controller and are not accessible directly. A transfer always begins
with \hex{00} in the lower bits and copies exactly 160 bytes, so the lower bits
are never in the \hexrange{A0}{FF} range.

Writing to the DMA register updates the upper bits of the DMA source address
and also triggers an OAM DMA transfer request, although the DMA transfer does
not begin immediately.

\begin{register}[H]
  \caption{\hex{FF46} - DMA - OAM DMA control register}
  {
    \ttfamily
    \begin{tabularx}{\linewidth}{|X|X|X|X|X|X|X|X|}
      \hline
      R/W-x                           & R/W-x & R/W-x & R/W-x & R/W-x & R/W-x & R/W-x & R/W-x \\
      \hline
      \multicolumn{8}{|c|}{DMA<7:0>} \\
      \hline
      bit 7                           & 6     & 5     & 4     & 3     & 2     & 1     & bit 0 \\
      \hline
    \end{tabularx}{\parfillskip=0pt\par}
  }

  \begin{description}[leftmargin=5em, style=nextline]
    \item[bit 0]
      \textbf{DMA<7:0>}: OAM DMA source address\\
      Specifies the top 8 bits of the OAM DMA source address.

      Writing to this register requests an OAM DMA transfer, but it's just a
      request and the actual DMA transfer starts with a delay.

      Reading this register returns the value that was previously written to
      the register. The stored value is not cleared on reset, so the initial
      value before the first write is unknown and should not be relied on.
  \end{description}
\end{register}

\begin{warning}
  Avoid writing \hexrange{E0}{FF} to the DMA register, because some poorly
designed flash carts can trigger bus conflicts or other dangerous behaviour.
\end{warning}

\subsection{OAM DMA address decoding}

The OAM DMA controller uses a simplified address decoding scheme, which leads
to some addresses being unusable as source addresses. Unlike normal memory
accesses, OAM DMA transfers interpret all accesses in the \hexrange{A000}{FFFF}
range as external RAM transfers. For example, if the OAM DMA wants to read
\hex{FF00}, it will output \hex{FF00} on the external address bus and will
assert the external RAM chip select signal. The P1 register which is normally
at \hex{FF00} is not involved at all, because OAM DMA address decoding only
uses the external bus and the video RAM bus. Instead, the resulting behaviour
depends on several factors, including the connected cartridge. Some flash carts
are not prepared for this unexpected scenario, and a bus conflict or worse
behaviour can happen.

\begin{table}[H]
  \caption{OAM DMA address decoding scheme}
  \rmfamily
  \begin{tabularx}{\linewidth}{|L|L|L|}
    \hline
    DMA register value & Used bus      & Asserted chip select signal \\
    \hline
    \hexrange{00}{7F}  & external bus  & external ROM (A15)          \\
    \hline
    \hexrange{80}{9F}  & video RAM bus & video RAM (MCS)             \\
    \hline
    \hexrange{A0}{FF}  & external bus  & external RAM (CS)           \\
    \hline
  \end{tabularx}{\parfillskip=0pt\par}
\end{table}

\subsection{OAM DMA transfer timing}

TODO

\subsection{OAM DMA bus conflicts}

TODO

\end{document}
