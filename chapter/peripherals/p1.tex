%!TEX root = ../../gbctr.tex
%!TEX program = lualatex
\providecommand{\main}{../..}
\documentclass[\main/gbctr.tex]{subfiles}
\begin{document}

\chapter{Port P1 (Joypad, Super Game Boy communication)}

\begin{register}[H]
  \caption{\hex{FF00} - P1 - Joypad/Super Game Boy communication register}
  {
    \ttfamily
    \begin{tabularx}{\linewidth}{|X|X|X|X|X|X|X|X|}
      \hline
      U-1                     & U-1                     & W-0 & W-0 & R-x & R-x & R-x & R-x   \\
      \hline
      \cellcolor{LightGray} - & \cellcolor{LightGray} - & P15 & P14 & P13 & P12 & P11 & P10   \\
      \hline
      bit 7                   & 6                       & 5   & 4   & 3   & 2   & 1   & bit 0 \\
      \hline
    \end{tabularx}{\parfillskip=0pt\par}
  }

  \begin{description}[leftmargin=5em, style=nextline]
    \item[bit 7-6]
      \textbf{Unimplemented}: Read as \bit{1}
    \item[bit 5]
      \textbf{P15}:
    \item[bit 4]
      \textbf{P14}:
    \item[bit 3]
      \textbf{P13}:
    \item[bit 2]
      \textbf{P12}:
    \item[bit 1]
      \textbf{P11}:
    \item[bit 0]
      \textbf{P10}:
  \end{description}
\end{register}

\end{document}
